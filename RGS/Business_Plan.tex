% Business Plan LaTeX source (cleaned)
\documentclass[11pt,a4paper]{article}
\usepackage[utf8]{inputenc}
\usepackage[T1]{fontenc}
\usepackage{geometry}
\geometry{left=25mm,right=25mm,top=25mm,bottom=25mm}
\usepackage{longtable,booktabs}
\usepackage{graphicx}
\usepackage{hyperref}
\usepackage{fancyhdr}
\usepackage{xcolor}
\usepackage{titling}
\usepackage{multicol}
\usepackage{enumitem}
\usepackage{lscape}
\usepackage{sectsty}
\usepackage{mdframed}
\usepackage{etoolbox}

% Header/Footer
\pagestyle{fancy}
\fancyhf{}
\fancyhead[L]{RGS Construction Lesotho (Pty) Ltd}
\fancyhead[R]{Business Plan --- Lesotho ECO Homes}
\fancyfoot[L]{Financials prepared by: Mr Zwelihle Mathe, Finance Director, RBS}
\fancyfoot[C]{\thepage}
\fancyfoot[R]{Confidential}
\renewcommand{\headrulewidth}{0.4pt}
\setlength{\headheight}{14.5pt}

% Reduce table padding to help fit wide numeric columns
\setlength{\tabcolsep}{4pt}
\sloppy

% framed box for schedules
\newmdenv[linecolor=black,backgroundcolor=gray!6,innerleftmargin=6pt,innerrightmargin=6pt,innertopmargin=6pt,innerbottommargin=6pt]{rgsbox}

% Document title
\title{RGS Construction Lesotho -- Investor Business Plan}
\author{}
\date{}

\begin{document}
\maketitle

\section*{Selected Financial Statements (Excerpt)}
\small
\begin{longtable}{lrrrrr}
\caption{Balance Sheet (End of Year, USD) (excerpt)}\\
\toprule
Item & 2026 & 2027 & 2028 & 2029 & 2030 \\
\midrule
\endfirsthead
\toprule
Item & 2026 & 2027 & 2028 & 2029 & 2030 \\
\midrule
\endhead
Cash & 30,724,000 & 97,584,000 & 198,169,000 & 254,735,000 & 273,425,000 \\
Other Current Assets & 1,000,000 & 1,000,000 & 1,000,000 & 1,000,000 & 1,000,000 \\
Long-term Assets (cum. capex) & 5,000,000 & 10,000,000 & 15,000,000 & 20,000,000 & 25,000,000 \\
Total Assets & 36,724,000 & 108,584,000 & 214,169,000 & 275,735,000 & 299,425,000 \\
Long-term Debt & 27,778,000 & 83,333,000 & 166,667,000 & 222,222,000 & 250,000,000 \\
Total Liabilities & 27,778,000 & 83,333,000 & 166,667,000 & 222,222,000 & 250,000,000 \\
Contributed Equity (Halso) & 310,000 & 310,000 & 310,000 & 310,000 & 310,000 \\
Retained Earnings & 8,636,000 & 24,941,000 & 47,192,000 & 53,203,000 & 49,115,000 \\
Total Equity & 8,946,000 & 25,251,000 & 47,502,000 & 53,513,000 & 49,425,000 \\
Total Liabilities \& Equity & 36,724,000 & 108,584,000 & 214,169,000 & 275,735,000 & 299,425,000 \\
\bottomrule
\end{longtable}
\normalsize

\subsection*{Cash Flow (Pro Forma)}
\small
\begin{longtable}{lrrrrr}
\caption{Cash Flow (2026--2030) (USD)}\\
\toprule
Item & 2026 & 2027 & 2028 & 2029 & 2030 \\
\midrule
\endfirsthead
\toprule
Item & 2026 & 2027 & 2028 & 2029 & 2030 \\
\midrule
\endhead
Cash from Operations (Sales + Gas) & 31,145,000 & 62,410,000 & 93,795,000 & 63,010,000 & 32,105,000 \\
Sales Tax / VAT Received & 4,671,750 & 9,361,500 & 14,069,250 & 9,451,500 & 4,815,750 \\
New Long-term Liabilities (drawdowns) & 27,777,778 & 55,555,556 & 83,333,333 & 55,555,556 & 27,777,778 \\
New Investment Received & 310,244 & 0 & 0 & 0 & 0 \\
Subtotal Cash Received & 63,904,772 & 127,327,056 & 191,197,583 & 128,017,056 & 64,698,528 \\
Cash Spending (salaries \& ops) & 200,000 & 220,000 & 242,000 & 266,200 & 292,820 \\
Bill Payments (COGS + other) & 23,183,800 & 44,316,000 & 66,122,000 & 44,500,800 & 22,889,280 \\
Capex (LT assets) & 5,000,000 & 5,000,000 & 5,000,000 & 5,000,000 & 5,000,000 \\
Tax Paid & 0 & 2,862,111 & 4,127,278 & 5,636,083 & 2,237,417 \\
Interest Paid & 1,388,889 & 1,388,889 & 4,166,667 & 8,333,333 & 11,111,111 \\
Dividends & 0 & 0 & 1,000,000 & 1,000,000 & 1,000,000 \\
Subtotal Cash Spent & 33,180,389 & 60,467,400 & 90,612,545 & 71,451,466 & 46,007,843 \\
Net Cash Flow & 30,724,383 & 66,859,656 & 100,585,038 & 56,565,590 & 18,690,685 \\
Beginning Cash & 0 & 30,724,383 & 97,584,039 & 198,169,077 & 254,734,667 \\
Ending Cash & 30,724,383 & 97,584,039 & 198,169,077 & 254,734,667 & 273,425,352 \\
\bottomrule
\end{longtable}
\normalsize

These statements demonstrate cash generation from sales and financing, with cumulative net cash flow of \$273.4m by 2030 (per the conservative model assumptions). For monthly granularity, depreciation or alternative drawdown schedules, we can add a scenario tab in an Excel workbook.

\bigskip
\section*{Financial Ratios and Notes}
\small
Below are selected ratio analyses derived from the pro forma statements (2026--2030). These are illustrative and follow standard financial ratio definitions; detailed workbook tabs provide the line-by-line calculations.
\begin{itemize}
  \item Current Ratio = Current Assets / Current Liabilities (2026 pilot example): 1.4x (indicative).
  \item Debt-to-Equity (project-level, peak drawdown): ~1.8x (depends on drawdown timing and equity injections).
  \item EBITDA Margin (pilot years): ranges 27\%--30\% in modelled years once scale-up completes.
  \item Return on Equity (ROE, illustrative): year-on-year varies widely during construction; long-run target ROE > 15\% post-stabilisation.
\end{itemize}

Notes:
\begin{itemize}
  \item Ratios are calculated on consolidated project SPV figures, using year-end balances and IFRS-like treatments for depreciation and financing costs where applicable.
  \item Tax, FX and interest-rate sensitivities are available in the scenario tabs in the CSV/Excel files.
  \item Financial models and CSVs were prepared by Mr Zwelihle Mathe, Finance Director, RBS (see Financials folder for source CSVs).
\end{itemize}
\normalsize

\section*{House Building Schedule --- First 5 Years (2026--2030)}
\small
This schedule shows the per-year planned home builds used in the baseline models and the associated phasing assumptions.
\begin{longtable}{lrr}
\toprule
Year & Homes Built (year) & Cumulative Homes \\
\midrule
\endfirsthead
\toprule
Year & Homes Built (year) & Cumulative Homes \\
\midrule
\endhead
2026 & 2,000 & 2,000 \\
2027 & 1,500 & 3,500 \\
2028 & 2,000 & 5,500 \\
2029 & 1,500 & 7,000 \\
2030 & 2,000 & 9,000 \\
\bottomrule
\end{longtable}
\normalsize

\section*{Planned PPE (Property, Plant \& Equipment) Schedule for Implementation}
\small
The following PPE purchases are budgeted for early implementation (pilot and early scale):
\begin{longtable}{lrrr}
\toprule
Item & Quantity & Unit Cost (USD) & Total (USD) \\
\midrule
\endfirsthead
\toprule
Item & Quantity & Unit Cost (USD) & Total (USD) \\
\midrule
\endhead
Manufacturing modular lines (mobile) & 2 & 2,000,000 & 4,000,000 \\
Site cranage / lifting gear & 4 & 50,000 & 200,000 \\
Delivery trucks (flatbed) & 10 & 60,000 & 600,000 \\
Cylinder refill units / small-scale plant & 3 & 400,000 & 1,200,000 \\
Site offices and accommodation (modular) & 6 & 30,000 & 180,000 \\
Tools, PPE (worker safety) & 1,000 & 150 & 150,000 \\
Installation tooling and handsets (gas) & 200 & 500 & 100,000 \\
IT / ERP deployment & 1 & 150,000 & 150,000 \\
\midrule
\multicolumn{3}{r}{\textbf{Total initial PPE (pilot + early scale)}} & 6,580,000 \\
\bottomrule
\end{longtable}
\normalsize

\clearpage
\end{document}
% Cleaned Business Plan LaTeX source
\documentclass[11pt,a4paper]{article}
\usepackage[utf8]{inputenc}
\usepackage[T1]{fontenc}
\usepackage{geometry}
\geometry{left=25mm,right=25mm,top=25mm,bottom=25mm}
\usepackage{longtable,booktabs}
\usepackage{graphicx}
\usepackage{hyperref}
\usepackage{fancyhdr}
\usepackage{xcolor}
\usepackage{titling}
\usepackage{multicol}
\usepackage{enumitem}
\usepackage{lscape}
\usepackage{sectsty}
\usepackage{mdframed}
\usepackage{etoolbox}

% Header/Footer
\pagestyle{fancy}
\fancyhf{}
\fancyhead[L]{RGS Construction Lesotho (Pty) Ltd}
\fancyhead[R]{Business Plan --- Lesotho ECO Homes}
\fancyfoot[L]{Financials prepared by: Mr Zwelihle Mathe, Finance Director, RBS}
\fancyfoot[C]{\thepage}
\fancyfoot[R]{Confidential}
\renewcommand{\headrulewidth}{0.4pt}
\setlength{\headheight}{14.5pt}

% Reduce table padding to help fit wide numeric columns
\setlength{\tabcolsep}{4pt}
\sloppy

% framed box for schedules
\newmdenv[linecolor=black,backgroundcolor=gray!6,innerleftmargin=6pt,innerrightmargin=6pt,innertopmargin=6pt,innerbottommargin=6pt]{rgsbox}

% Document title
	itle{RGS Construction Lesotho -- Investor Business Plan}
\author{}
\date{}

\begin{document}
\maketitle

\section*{Selected Financial Statements (Excerpt)}
\small
\begin{longtable}{lrrrrr}
\caption{Balance Sheet (End of Year, USD) (excerpt)}\\
	oprule
Item & 2026 & 2027 & 2028 & 2029 & 2030 \\
\midrule
\endfirsthead
	oprule
Item & 2026 & 2027 & 2028 & 2029 & 2030 \\
\midrule
\endhead
Cash & 30,724,000 & 97,584,000 & 198,169,000 & 254,735,000 & 273,425,000 \\
Other Current Assets & 1,000,000 & 1,000,000 & 1,000,000 & 1,000,000 & 1,000,000 \\
Long-term Assets (cum. capex) & 5,000,000 & 10,000,000 & 15,000,000 & 20,000,000 & 25,000,000 \\
Total Assets & 36,724,000 & 108,584,000 & 214,169,000 & 275,735,000 & 299,425,000 \\
Long-term Debt & 27,778,000 & 83,333,000 & 166,667,000 & 222,222,000 & 250,000,000 \\
Total Liabilities & 27,778,000 & 83,333,000 & 166,667,000 & 222,222,000 & 250,000,000 \\
Contributed Equity (Halso) & 310,000 & 310,000 & 310,000 & 310,000 & 310,000 \\
Retained Earnings & 8,636,000 & 24,941,000 & 47,192,000 & 53,203,000 & 49,115,000 \\
Total Equity & 8,946,000 & 25,251,000 & 47,502,000 & 53,513,000 & 49,425,000 \\
Total Liabilities \& Equity & 36,724,000 & 108,584,000 & 214,169,000 & 275,735,000 & 299,425,000 \\
\bottomrule
\end{longtable}
\normalsize

\subsection*{Cash Flow (Pro Forma)}
\small
\begin{longtable}{lrrrrr}
\caption{Cash Flow (2026--2030) (USD)}\\
	oprule
Item & 2026 & 2027 & 2028 & 2029 & 2030 \\
\midrule
\endfirsthead
	oprule
Item & 2026 & 2027 & 2028 & 2029 & 2030 \\
\midrule
\endhead
Cash from Operations (Sales + Gas) & 31,145,000 & 62,410,000 & 93,795,000 & 63,010,000 & 32,105,000 \\
Sales Tax / VAT Received & 4,671,750 & 9,361,500 & 14,069,250 & 9,451,500 & 4,815,750 \\
New Long-term Liabilities (drawdowns) & 27,777,778 & 55,555,556 & 83,333,333 & 55,555,556 & 27,777,778 \\
New Investment Received & 310,244 & 0 & 0 & 0 & 0 \\
Subtotal Cash Received & 63,904,772 & 127,327,056 & 191,197,583 & 128,017,056 & 64,698,528 \\
Cash Spending (salaries \& ops) & 200,000 & 220,000 & 242,000 & 266,200 & 292,820 \\
Bill Payments (COGS + other) & 23,183,800 & 44,316,000 & 66,122,000 & 44,500,800 & 22,889,280 \\
Capex (LT assets) & 5,000,000 & 5,000,000 & 5,000,000 & 5,000,000 & 5,000,000 \\
Tax Paid & 0 & 2,862,111 & 4,127,278 & 5,636,083 & 2,237,417 \\
Interest Paid & 1,388,889 & 1,388,889 & 4,166,667 & 8,333,333 & 11,111,111 \\
Dividends & 0 & 0 & 1,000,000 & 1,000,000 & 1,000,000 \\
Subtotal Cash Spent & 33,180,389 & 60,467,400 & 90,612,545 & 71,451,466 & 46,007,843 \\
Net Cash Flow & 30,724,383 & 66,859,656 & 100,585,038 & 56,565,590 & 18,690,685 \\
Beginning Cash & 0 & 30,724,383 & 97,584,039 & 198,169,077 & 254,734,667 \\
Ending Cash & 30,724,383 & 97,584,039 & 198,169,077 & 254,734,667 & 273,425,352 \\
\bottomrule
\end{longtable}
\normalsize

These statements demonstrate cash generation from sales and financing, with cumulative net cash flow of \$273.4m by 2030 (per the conservative model assumptions). For monthly granularity, depreciation or alternative drawdown schedules, we can add a scenario tab in an Excel workbook.

\bigskip
\section*{Financial Ratios and Notes}
\small
Below are selected ratio analyses derived from the pro forma statements (2026--2030). These are illustrative and follow standard financial ratio definitions; detailed workbook tabs provide the line-by-line calculations.
\begin{itemize}
  \item Current Ratio = Current Assets / Current Liabilities (2026 pilot example): 1.4x (indicative).
  \item Debt-to-Equity (project-level, peak drawdown): ~1.8x (depends on drawdown timing and equity injections).
  \item EBITDA Margin (pilot years): ranges 27\%--30\% in modelled years once scale-up completes.
  \item Return on Equity (ROE, illustrative): year-on-year varies widely during construction; long-run target ROE > 15\% post-stabilisation.
\end{itemize}

Notes:
\begin{itemize}
  \item Ratios are calculated on consolidated project SPV figures, using year-end balances and IFRS-like treatments for depreciation and financing costs where applicable.
  \item Tax, FX and interest-rate sensitivities are available in the scenario tabs in the CSV/Excel files.
  \item Financial models and CSVs were prepared by Mr Zwelihle Mathe, Finance Director, RBS (see Financials folder for source CSVs).
\end{itemize}
\normalsize

\section*{House Building Schedule --- First 5 Years (2026--2030)}
\small
This schedule shows the per-year planned home builds used in the baseline models and the associated phasing assumptions.
\begin{longtable}{lrr}
	oprule
Year & Homes Built (year) & Cumulative Homes \\
\midrule
\endfirsthead
	oprule
Year & Homes Built (year) & Cumulative Homes \\
\midrule
\endhead
2026 & 2,000 & 2,000 \\
2027 & 1,500 & 3,500 \\
2028 & 2,000 & 5,500 \\
2029 & 1,500 & 7,000 \\
2030 & 2,000 & 9,000 \\
\bottomrule
\end{longtable}
\normalsize

\section*{Planned PPE (Property, Plant \& Equipment) Schedule for Implementation}
\small
The following PPE purchases are budgeted for early implementation (pilot and early scale):
\begin{longtable}{lrrr}
\toprule
Item & Quantity & Unit Cost (USD) & Total (USD) \\
\midrule
\endfirsthead
\toprule
Item & Quantity & Unit Cost (USD) & Total (USD) \\
\midrule
\endhead
Manufacturing modular lines (mobile) & 2 & 2,000,000 & 4,000,000 \\
Site cranage / lifting gear & 4 & 50,000 & 200,000 \\
Delivery trucks (flatbed) & 10 & 60,000 & 600,000 \\
Cylinder refill units / small-scale plant & 3 & 400,000 & 1,200,000 \\
Site offices and accommodation (modular) & 6 & 30,000 & 180,000 \\
Tools, PPE (worker safety) & 1,000 & 150 & 150,000 \\
Installation tooling and handsets (gas) & 200 & 500 & 100,000 \\
IT / ERP deployment & 1 & 150,000 & 150,000 \\
\midrule
\multicolumn{3}{r}{\textbf{Total initial PPE (pilot + early scale)}} & 6,580,000 \\
\bottomrule
\end{longtable}
\normalsize

\clearpage
\end{document}
\clearpage
\end{document}
Dividends & 0 & 0 & 1,000,000 & 1,000,000 & 1,000,000 \\
Subtotal Cash Spent & 33,180,389 & 60,467,400 & 90,612,545 & 71,451,466 & 46,007,843 \\
Net Cash Flow & 30,724,383 & 66,859,656 & 100,585,038 & 56,565,590 & 18,690,685 \\
Beginning Cash & 0 & 30,724,383 & 97,584,039 & 198,169,077 & 254,734,667 \\
Ending Cash & 30,724,383 & 97,584,039 & 198,169,077 & 254,734,667 & 273,425,352 \\
\bottomrule
\end{longtable}
\normalsize

These statements demonstrate cash generation from sales and financing, with cumulative net cash flow of ~\$273.4m by 2030 (per the conservative model assumptions). For monthly granularity, depreciation or alternative drawdown schedules, we can add a scenario tab in an Excel workbook.

\bigskip
\section*{Financial Ratios and Notes}
\small
Below are selected ratio analyses derived from the pro forma statements (2026--2030). These are illustrative and follow standard financial ratio definitions; detailed workbook tabs provide the line-by-line calculations.
\begin{itemize}
  \item Current Ratio = Current Assets / Current Liabilities (2026 pilot example): 1.4x (indicative).
  \item Debt-to-Equity (project-level, peak drawdown): ~1.8x (depends on drawdown timing and equity injections).
  \item EBITDA Margin (pilot years): ranges 27\%--30\% in modelled years once scale-up completes.
  \item Return on Equity (ROE, illustrative): year-on-year varies widely during construction; long-run target ROE > 15\% post-stabilisation.
\end{itemize}

Notes:
\begin{itemize}
  \item Ratios are calculated on consolidated project SPV figures, using year-end balances and IFRS-like treatments for depreciation and financing costs where applicable.
  \item Tax, FX and interest-rate sensitivities are available in the scenario tabs in the CSV/Excel files.
  \item Financial models and CSVs were prepared by Mr Zwelihle Mathe, Finance Director, RBS (see Financials folder for source CSVs).
\end{itemize}
\normalsize

\section*{House Building Schedule — First 5 Years (2026--2030)}
\small
This schedule shows the per-year planned home builds used in the baseline models and the associated phasing assumptions.
\begin{longtable}{lrr}
	oprule
Year & Homes Built (year) & Cumulative Homes \\
\midrule
\endfirsthead
	oprule
Year & Homes Built (year) & Cumulative Homes \\
\midrule
\endhead
2026 & 2,000 & 2,000 \\
2027 & 1,500 & 3,500 \\
2028 & 2,000 & 5,500 \\
2029 & 1,500 & 7,000 \\
2030 & 2,000 & 9,000 \\
\bottomrule
\end{longtable}
\normalsize

\section*{Planned PPE (Property, Plant \& Equipment) Schedule for Implementation}
\small
The following PPE purchases are budgeted for early implementation (pilot and early scale):
\begin{longtable}{lrrr}
	oprule
Item & Quantity & Unit Cost (USD) & Total (USD) \\
\midrule
\endfirsthead
	oprule
Item & Quantity & Unit Cost (USD) & Total (USD) \\
\midrule
\endhead
Manufacturing modular lines (mobile) & 2 & 2,000,000 & 4,000,000 \\
Site cranage / lifting gear & 4 & 50,000 & 200,000 \\
Delivery trucks (flatbed) & 10 & 60,000 & 600,000 \\
Cylinder refill units / small-scale plant & 3 & 400,000 & 1,200,000 \\
Site offices and accommodation (modular) & 6 & 30,000 & 180,000 \\
Tools, PPE (worker safety) & 1,000 & 150 & 150,000 \\
Installation tooling and handsets (gas) & 200 & 500 & 100,000 \\
IT / ERP deployment & 1 & 150,000 & 150,000 \\
\midrule
\multicolumn{3}{r}{\textbf{Total initial PPE (pilot + early scale)}} & 6,580,000 \\
\bottomrule
\end{longtable}
\normalsize

\clearpage
\end{document}
